\documentclass[british,10pt]{beamer}
\usepackage{pifont}
\usepackage{listings}
\usepackage{adjustbox} % for \adjincludegraphics

% Beamer specific settings

% Other styles can be used instead of Boadilla, e.g., to add an index column.
\mode<presentation>
{
  \usetheme{Boadilla} % Without index, with footer
}

\title{Building a better VHDL testing environment}
\author[J. Guillaume]{Joren Guillaume}
\date[JAN'15, Gent]{Thesis presentation}
\institute[Ghent University]
{
  FEA\\
  Ghent University
}

% This command makes the logo appear at the right side which does not fit
% the UGent style. A solution is found further below.
% \logo{\includegraphics[scale=0.25]{logolabel.jpg}}

\AtBeginSection[]
{
  \begin{frame}<beamer>\frametitle{Outline}
    \tableofcontents[currentsection,currentsubsection]
  \end{frame}
}

\setbeamersize{text margin left=1cm}
\setbeamersize{text margin right=1cm}

\begin{document}


% Titlepage containing the logo of the Faculty (Engineering in this example)
\begin{frame}[plain]
\mode<presentation>{\includegraphics[width=\textwidth]{tw.pdf}}
  \titlepage
\end{frame}


% UGent logo at left side from second slide on.
\setbeamertemplate{sidebar left}{ 
\vfill 
\rlap{%\hskip0.1cm 
 \includegraphics[scale=0.3]{logolabel2.jpg} } 
\vskip10pt 
}

\begin{frame}<beamer>\frametitle{Outline}
  \tableofcontents
\end{frame}

\section{Situating}
\subsection{VHDL}
\begin{frame}\frametitle{VHDL}
\begin{itemize}
\item VHSIC Hardware Description Language
\item Used for describing digital and mixed-signal systems 
\item Developed by U.S. Department of Defense
\begin{itemize}
\item Document \ding{222} Simulate \ding{222} Synthesize
\end{itemize}
\end{itemize}
\end{frame}

\subsection{Testing and problems}

\begin{frame}\frametitle{Testing VHDL}
\begin{columns}
\begin{column}{0.05\textwidth}
\end{column}
\begin{column}{0.55\textwidth}
Test benches
\begin{itemize}
\item Unit Under Test (UUT)
\item Apply stimuli
\item Signal/output tracking
\begin{itemize}
\item Assertions
\item Comparison to desired result
\item Wave-check
\end{itemize}
\end{itemize}
%\vspace{0.25cm}
%Problems
%\begin{itemize}
%\item Non-standardized process
%\item Single point of failure
%\item Time consuming
%\end{itemize}
\end{column}
\column{0.4\textwidth}
\raggedleft
\includegraphics[width=\textwidth]{images/TB.pdf}
\end{columns}
\end{frame}

\begin{frame}\frametitle{Testing VHDL}
Problems with testing
\begin{itemize}
\item Non-standardized process
\item Single point of failure
\item Time consuming
\end{itemize}
\end{frame}

\subsection{Software development techniques}
\begin{frame}\frametitle{Software development techniques}
\begin{columns}
\begin{column}{0.6\textwidth}
Unit testing
\begin{itemize}
\item Unit = smallest behaviour in code
\item Test failure \ding{222} exact location
\end{itemize}
\vskip3pt
Test First Development
\begin{itemize}
\item Create test before the code
\item How will the code behave?
\end{itemize}
\vskip3pt
Test Driven Development
\begin{itemize}
\item TFD \& refactoring
\item Proven to significantly reduce errors
\end{itemize}
\end{column}
\column{0.4\textwidth}
\includegraphics[width=\textwidth]{images/tdd.pdf}
\end{columns}
\end{frame}

\section{Proposed solution}

\begin{frame}\frametitle{Proposed solution}
Standardized testing framework
\begin{itemize}
\item Based on software development techniques
\item Cross platform
\item At the core: Python script
\item Utility library
\item Continuous Integration system
\end{itemize}
\end{frame}

%\begin{frame}\frametitle{VHDL - design flow}
%\includegraphics[width=\textwidth]{images/designflow2.pdf}
%\end{frame}

%\begin{frame}[fragile]\frametitle{Testing VHDL - assertions example }
%D-flipflop
%\begin{itemize}
%\item d: delay, input
%\item q: output
%\end{itemize}
%\centering
%\begin{lstlisting}[language=VHDL, tabsize=4, frame=single, framesep=2mm, belowskip=8pt, aboveskip=8pt, showstringspaces=false, basicstyle=\scriptsize]
%	...
%		assert q = '0'
%			report "Wrong output value at startup" severity FAILURE;
%		d <= '1';
%     	WAIT FOR clk_period;
%     	assert q = '1'
%			report "Wrong output value at first test" severity FAILURE;
%	...
%\end{lstlisting}
%\end{frame}

%\subsection{Software development techniques}

%\begin{frame}\frametitle{Unit testing}
%Unit testing
%\begin{itemize}
%\item Unit: smallest behaviour in code
%\item Unit test: Tests that behaviour
%\item Test success/failure \ding{222} excellent locator
%\end{itemize}
%\end{frame}

%\begin{frame}\frametitle{Unit testing}
%Unit testing
%\begin{itemize}
%\item Unit: smallest behaviour in code
%\item Unit test: Tests that behaviour
%\item Test success/failure \ding{222} excellent locator
%\end{itemize}
%\end{frame}
%
%\begin{frame}\frametitle{Test Driven Development}
%\begin{columns}
%\begin{column}{0.6\textwidth}
%Test Driven Development
%\begin{itemize}
%\item Software development technique
%\item Proven to significantly reduce errors
%\item All behaviour is tested
%\item Unit testing \& short development cycle
%\item Red - Green - Refactor
%\end{itemize}
%\end{column}
%\column{0.4\textwidth}
%\includegraphics[width=\textwidth]{images/tdd.pdf}
%\end{columns}
%\end{frame}

%\begin{frame}\frametitle{TDD - design flow}
%\includegraphics[width=\textwidth]{images/TDDflow2.pdf}
%\end{frame}

%\section{Proposed solution}
\subsection{VHDL testing framework}

\begin{frame}\frametitle{VHDL testing framework}
%\vskip20pt
\begin{columns}
\begin{column}{0.60\textwidth}
\begin{enumerate}
\item Split test bench into groups of tests
\item Separate groups into new test benches
\item Compile sources and new test benches
\item Execute test benches
\item Capture and process results
\end{enumerate}
\end{column}
\column{0.05\textwidth}
\vskip12pt
\Huge\}
\vskip1pt
\}
\column{0.30\textwidth}
\ding{222} Prepare
\vskip13pt
\ding{222} Process
\vskip17pt
\ding{222} Execute
\end{columns}
%\vskip20pt
%\centering
%\includegraphics[width=.7\textwidth]{images/ppe.pdf}
\end{frame}

\subsection{Using the framework}
\begin{frame}\frametitle{Preparing test benches}
\vskip50pt
\begin{columns}
\begin{column}{0.6\textwidth}
\begin{itemize}
\item Use Bitvis utility library
\begin{itemize}
\item Faster coding
\item Improved readability
\end{itemize}
\item Separate independent tests
\begin{itemize}
\item Line by line
\item Start/Stop
\item Partitioned
\end{itemize}
\item Create commands file
\end{itemize}
%\hskip24pt\ding{222} Unit testing
\end{column}
\column{0.4\textwidth}
\includegraphics[width=0.8\textwidth]{images/bitvis.png}
\end{columns}
\centering
\vskip50pt
\includegraphics[width=.7\textwidth]{images/ppe1.pdf}
\end{frame}

\begin{frame}[fragile]\frametitle{Modified test bench example}
D flip-flop
\begin{itemize}
\item Old test bench:
\end{itemize}
\begin{lstlisting}[language=VHDL, tabsize=4, frame=single, framesep=2mm, belowskip=5pt, aboveskip=5pt, showstringspaces=false, basicstyle=\scriptsize]
assert q = '0'
    report "Wrong output value at startup" severity FAILURE;
d <= '1';
WAIT FOR clk_period;
assert q = '1'
    report "Wrong output value at first test" severity FAILURE;
\end{lstlisting}
\vskip1pt
\begin{itemize}
\item Modified test bench:
\end{itemize}
\begin{lstlisting}[language=VHDL, tabsize=4, frame=single, framesep=2mm, belowskip=5pt, aboveskip=5pt, showstringspaces=false, basicstyle=\scriptsize]
--Test 1
    check_value(q = '0', FAILURE, "Wrong output value at startup");
    write(d, '1', "DFF");
    check_value(q = '1', FAILURE, "Wrong output value at first test");
    ...
--End 1
\end{lstlisting}
\end{frame}


\begin{frame}\frametitle{Processing and compiling}
\vskip20pt
Python script:
\begin{itemize}
\item Reads command line arguments
\item Reads modified test bench
\item Groups tests into new test benches
\end{itemize}
\vskip5pt
\centering
\includegraphics[width=.8\textwidth]{images/tbsplit.pdf}
\vskip20pt
\includegraphics[width=.7\textwidth]{images/ppe2.pdf}
\end{frame}

\begin{frame}\frametitle{Processing and compiling}
\vskip40pt
ModelSim:
\begin{itemize}
\item Compiles source code
\item Compiles test benches\\
\hskip2pt\ding{222} One entity, many architectures
\end{itemize}
\centering
\includegraphics[width=.75\textwidth]{images/sources.pdf}
\vskip10pt
\includegraphics[width=.7\textwidth]{images/ppe2.pdf}
\end{frame}


\begin{frame}\frametitle{Execution and results}
\vskip48pt
ModelSim:
\begin{itemize}
\item Executes each test bench
\end{itemize}
\vskip5pt
Python script:
\begin{itemize}
\item Captures ModelSim output
\item Processes results
\begin{itemize}
\item Text report
\item JUnit XML report
\end{itemize}
\end{itemize}
\centering
\vskip48pt
\includegraphics[width=.7\textwidth]{images/ppe3.pdf}
\end{frame}

\subsection{Automation}
\begin{frame}\frametitle{Automation}
\begin{columns}
\begin{column}{0.6\textwidth}
Hudson-CI
\begin{itemize}
\item Gets latest version from RC
\begin{itemize}
\item Timed retrieval
\item Detect changes
\end{itemize}
\item Automated script execution
\item Result progress (XML)
\end{itemize}
\end{column}
\column{0.4\textwidth}
\includegraphics[width=0.8\textwidth]{images/hudson.png}
\end{columns}
\end{frame}

%\subsection{Utility library}
%
%\begin{frame}\frametitle{Utility library}
%\begin{columns}
%\begin{column}{0.6\textwidth}
%Bitvis utility library
%\begin{itemize}
%\item Compatible with all VHDL versions
%\item Expands VHDL functions
%\begin{itemize}
%\item Easy value checking
%\item Formatted output
%\end{itemize}
%\item Quick \& uniform coding
%\begin{itemize}
%\item Reduces time spent coding
%\item Improves readability
%\end{itemize}
%\end{itemize}
%\end{column}
%\column{0.4\textwidth}
%\includegraphics[width=0.8\textwidth]{images/bitvis.png}
%\end{columns}
%\end{frame}

%\subsection{Continuous Integration}
%
%\begin{frame}\frametitle{Continuous Integration}
%\begin{columns}
%\begin{column}{0.6\textwidth}
%Hudson-CI
%\begin{itemize}
%\item Centralized, automated testing
%\begin{itemize}
%\item Revision control
%\item Timed building
%\end{itemize}
%\item Easy to read results
%\item Very customizable
%%\item Standardized test reports (XML)
%%\item Displays statistics
%\end{itemize}
%\end{column}
%\column{0.4\textwidth}
%\includegraphics[width=0.8\textwidth]{images/hudson.png}
%\end{columns}
%\end{frame}

%\subsection{Python script}
%
%\begin{frame}\frametitle{Python script}
%Test bench parser
%\begin{itemize}
%\item Separates independent tests
%\begin{itemize}
%\item Remove single point of failure
%\end{itemize}
%\item Customizable
%\begin{itemize}
%\item Command-line arguments
%\end{itemize}
%\item Multiple useful outputs
%\begin{itemize}
%\item Processed text-based report
%\item Processed XML report
%\end{itemize}
%\end{itemize}
%\end{frame}
%
%\begin{frame}\frametitle{Script workflow}
%\includegraphics[width=\textwidth]{images/parserwork.pdf}
%\newline{}
%\centering(Steps of the parser logged separately)
%\end{frame}


%\subsection{Using the framework}
%
%\begin{frame}\frametitle{Using the framework}
%Preparing the test bench
%\begin{enumerate}
%\item Import the utility library
%\item Decide on separation method
%\begin{itemize}
%\item Line by line \ding{222} No editing test bench
%\item Start/Stop
%\item Partitioned (recommended)
%\end{itemize}
%\item Create command file
%\end{enumerate}
%\end{frame}


%\begin{frame}[fragile]\frametitle{Modified test bench example}
%D flip-flop
%\begin{itemize}
%\item Old test bench:
%\end{itemize}
%\begin{lstlisting}[language=VHDL, tabsize=4, frame=single, framesep=2mm, belowskip=5pt, aboveskip=5pt, showstringspaces=false, basicstyle=\scriptsize]
%assert q = '0'
%    report "Wrong output value at startup" severity FAILURE;
%d <= '1';
%WAIT FOR clk_period;
%assert q = '1'
%    report "Wrong output value at first test" severity FAILURE;
%\end{lstlisting}
%\vskip1pt
%\begin{itemize}
%\item Modified test bench:
%\end{itemize}
%\begin{lstlisting}[language=VHDL, tabsize=4, frame=single, framesep=2mm, belowskip=5pt, aboveskip=5pt, showstringspaces=false, basicstyle=\scriptsize]
%--Test 1
%    check_value(q = '0', FAILURE, "Wrong output value at startup");
%    write(d, '1', "DFF");
%    check_value(q = '1', FAILURE, "Wrong output value at first test");
%    ...
%--End 1
%\end{lstlisting}
%\end{frame}


%\begin{frame}[fragile]\frametitle{Using the framework}
%Running Hudson-CI
%\begin{enumerate}
%\item Create new job
%\item Optional: set for import from revision control source
%\item Set correct parser flags in shell command
%\item Build \& check results
%\end{enumerate}
%\vskip5pt
%\begin{lstlisting}[language=bash, tabsize=4, frame=single, framesep=2mm, belowskip=8pt, aboveskip=8pt, showstringspaces=false, basicstyle=\scriptsize]
%python src\testbench_parser.py -m partitioned -l sim\tb_dff_r.vhd
%\end{lstlisting}
%\end{frame}

%\begin{frame}[fragile]\frametitle{Job example}
%\begin{lstlisting}[language=bash, tabsize=4, frame=single, framesep=2mm, belowskip=8pt, aboveskip=8pt, showstringspaces=false, basicstyle=\scriptsize]
%$ python src\testbench_parser.py -m partitioned -l sim\tb_dff_r.vhd
%\end{lstlisting}
%\end{frame}


%\begin{frame}\frametitle{Framework design flow}
%\centering
%\includegraphics[width=\textwidth]{images/parserwork.pdf}
%\end{frame}

\section{Concluding}

\subsection{Results}
\begin{frame}\frametitle{Results}
Multiple open-source projects tested
\vskip10pt
{\centering
\includegraphics[width=\textwidth]{images/jobs.png}}
%\\\hskip10pt\ding{222} Bypassed single point of failure
%\\\hskip10pt\ding{222} Automated testing = faster development
%\\\hskip10pt\ding{222} Successful runs when VHDL code OK
%\\\hskip10pt\ding{222} Partially completed test-runs even with faults in code
%\\\hskip10pt\ding{222} Unsuccessful runs only due to compilation errors\\
\end{frame}


\begin{frame}\frametitle{Results}
Precise debugging information
{\centering
\includegraphics[width=.95\textwidth]{images/results1.png}}
\end{frame}
%
%\begin{frame}\frametitle{Results}
%{\centering
%\includegraphics[width=.85\textwidth]{images/results2.png}}
%\end{frame}
%%\item Code duplication increases compile \& simulation time

\subsection{Future work}

\begin{frame}\frametitle{Future work}
\begin{itemize}
\item Wider, better tool support
\item Lexical analysis
\begin{itemize}
\item Automated partitioning
\item Smart test bench generation
\end{itemize}
\item Adapted CI tool
\begin{itemize}
\item Specific needs of hardware development
\end{itemize}
\end{itemize}
\end{frame}
%%VOLGENDE
%%STAP
%%IN
%%DEZE
%%ONTWIKKELING
%%WAT
%%NU?

\subsection{Conclusion}

\begin{frame}\frametitle{Conclusion}
\begin{itemize}
\item Software methods are applicable if:
\begin{itemize}
\item Tailored to development needs
\item Integrated with existing methods
\end{itemize}
\vskip3pt
\item The framework provided:
\begin{itemize}
\item Easier to read code
\item Precise debugging information
\item Eliminated single point of failure
\end{itemize}
\end{itemize}
\end{frame}


\begin{frame}\frametitle{End}
\centering
\Large
Thanks for your attention!
\vskip20pt
Questions?
\end{frame}

%\begin{frame}\frametitle{Commentaries on developing VHDL}
%\begin{itemize}
%\item Outdated practices
%\begin{itemize}
%\item An industry stuck in 1993
%\item "Don't fix what isn't broken"
%\end{itemize}
%\item Hardware engineers are not software developers
%\begin{itemize}
%\item Little to no software development experience
%\item Taught by seniors at work
%\end{itemize}
%\item VHDL has no reflection or introspection
%\begin{itemize}
%\item Could make test benches even more compact
%\end{itemize}
%\end{itemize}
%\end{frame}
%
%\subsection{Conclusion}
%
%\begin{frame}\frametitle{Conclusion}
%\begin{itemize}
%\item Software methods are applicable to an extent if:
%\begin{itemize}
%\item Tailored to development needs
%\item Integrated with existing methods
%\end{itemize}
%\item Framework provided:
%\begin{itemize}
%\item Easier to read code
%\item Precise debugging information
%\item Faster development
%\end{itemize}
%\end{itemize}
%\end{frame}

\section{Demo}
\begin{frame}\frametitle{Demo}
\centering
\Huge Demo
\end{frame}

\end{document}
